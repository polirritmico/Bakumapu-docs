%%%%%%%%%%%%%%%%%%%%%%%%%%%%%%%%%%%%%%%%%%%%%%%%%%%%%%%%%%
%% Flujo de despliegue

\section{Integración y Distribución continua (CD/CI)}\label{pipeline:entrega-continua}

\subsection{Acerca del Pipeline o Flujo de entrega}\label{pipeline:acerca-del-pipeline}

Entendemos como pipeline (tubería) el conjunto de herramientas y procedimientos necesarios para transformar un commit de código a un output ejecutable que cumpla todos nuestros criterios de publicación. Es una cadena de producción automatizada ---como la de una fábrica--- que determina si el software se encuentra o no preparado para un lanzamiento, ajusta y empaqueta todas las dependencias necesarias del proyecto y genera los binarios para su distribución a los usuarios.

En esta cadena se ejecutarán de forma automática test unitarios, de integración, de rendimiento y todos los necesarios para determinar el estado actual del código en el repositorio. Por ejemplo, para comprobar mínimos de rendimiento se podría incorporar a nuestra pipeline un test de estrés para validar \lsc{FPS} mínimos.

Además de las pruebas, el sistema integrará todas las dependencias necesarias y empaquetará los binarios y cualquier recurso necesario para la distribución del programa a los sistemas de los clientes (win64, linux, android).

A nivel de gestión la pipeline nos permite determinar desde el primer momento los criterios de lanzamiento y organizar todo el flujo de trabajo en esa dirección evitando sorpresas y problemas de integración innecesarios. El tener un set herramientas y procedimientos para asegurar lanzamientos estables y funcionales desde el principio del proyecto es un activo fundamental para Bakumapu.

\subsubsection{Eficiencia}\label{pipeline:eficiencia}

Un factor muy importante con respecto al pipeline es la eficiencia total del sistema. Ya que queremos ser capaces de ejecutar el proceso múltiples veces al día es necesario \emph{optimizar el sistema para que no tarde más de 1 hora} de principio a fin (de commit a deploy). Este criterio de 1 hora otorgará márgen suficiente para que se pueda corregir o revertir cualquier commit que genere problemas; o al menos nos dará la información necesaria para su revisión o estudio.

\subsubsection{Repositorio de artefactos}

Entendemos como artefactos todos los binarios, librerías, resources y archivos necesarios para ejecutar correctamente el programa en un sistema operativo específico.

Dado que el uso de \lsc{GIT} trae complicaciones innecesarias de manera temporal se utilizará una carpeta compartida de Google Drive de manera temporal mientras no sea necesaria una solución más robusta.

\href{https://drive.google.com/drive/folders/15OtN9fs-UASOKTRcWGbR-knWMSmQGyO_?usp=sharing}

\subsection{Etapas de la pipeline}




\subsection{Pipeline}

\begin{enumerate}
  \item Merge hacia la rama \textbf{deploy}.
  \item Test unitarios, de integración y static analysis.
  \item Builds Windows, Android y Linux.
  \item Test de rendimiento
  \item Test de aceptación
  \item Test de sistema (instalación y ejecución)
  \item Deploy a las plataformas de distribución.
\end{enumerate}


% =============================================================================

\subsection{Tipos de Test}\label{pipeline:tipos-de-test}

\begin{itemize}
  \item \textbf{Unit Test:} Test sencillos y específicos hechos para probar
      funciones o partes puntuales del código. Deberían correr en memoria y no
      llamar entidades externas como bases de datos, sistema de archivos,
      acceder a redes, etc. Lo importante es probar la unidad de código y no
      sus dependencias. \item \textbf{Integration Test:} Test que combinan
      secciones unitarias de código para comprobar que la interacción de sus
      distintos métodos funcionan correctamente.
  \item \textbf{Acceptance Test:} Se prueba el software y el sistema al
      completo desde el punto de vista de un agente externo al desarrollo
      revisando que el programa satisface todas las especificaciones del
      usuario. \item \textbf{Performance Test:} Se revisa que el software
      cumpla tiempos de respuesta, estabilidad y \lsc{FPS}.
  \item \textbf{Static Analysis:} 
  \item \textbf{Sign-Offs: Regulaciones:} 
  \item \textbf{Security Test:} 
  \item \textbf{Scalability Test:} 
\end{itemize}


Exportar a Linux, Windows y Android:

Linux:
\begin{lstlisting}
$ godot --no-window --export-debug "Linux" export/linux/Bakumapu.x86_64
\end{lstlisting}

Windows:
\begin{lstlisting}
$ godot --no-window --export-debug "Windows" export/windows/Bakumapu.exe
\end{lstlisting}

Android:
\begin{lstlisting}
$ godot --no-window --export-debug "Android" export/android/Bakumapu.apk
\end{lstlisting}
