%%%%%%%%%%%%%%%%%%%%%%%%%%%%%%%%%%%%%%%%%%%%%%%%%%%%%%%%%%
%% Flujo de despliegue

\section{Integración y Distribución continua (CD/CI)}\label{pipeline:entrega-continua}

\subsection{Pipeline o Flujo de entrega}\label{pipeline:flujo-de-entrega}

Entendemos como pipeline (tubería) el conjunto de herramientas y procedimientos
necesarios para transformar un commit de código a un output ejecutable
completamente preparado para su publicación. Es una cadena de producción
automatizada que determina en base a criterios preestablecidos si el software
se encuentra o no preparado para un lanzamiento, ajusta y empaqueta todas las
dependencias necesarias del proyecto y empaqueta los binarios para su
distribución a usuarios.

En esta cadena se ejecutarán de forma automática test unitarios, de
integración, de rendimiento y todos los necesarios para determinar el estado
actual del código del repositorio. Por ejemplo, para comprobar mínimos de
rendimiento se podría incorporar a nuestra pipeline un test de estrés para
validar \lsc{fps} mínimos.

Además de las pruebas, el sistema integrará todas las dependencias necesarias y
empaquetará los binarios y cualquier recurso necesario para la distribución del
programa a los sistemas de los clientes (win64, linux, osx).

A nivel de gestión el sistema nos permite determinar desde el primer momento
los criterios de lanzamiento y organizar todo el flujo de trabajo en esa
dirección.

\subsection{Eficiencia}\label{pipeline:eficiencia}

Un factor muy importante con respecto a nuestro pipeline es la eficiencia total
del sistema. Ya que queremos ser capaces de ejecutar el proceso múltiples veces
al día, es necesario \emph{optimizar el sistema para que no tarde más de 1
hora} de principio a fin (de commit a deploy).

Este criterio de 1 hora otorgará márgen suficiente para que se puede corregir o
revertir cualquier commit que genere problemas; o al menos nos dará la
información necesaria para su revisión o estudio.

\subsection{Repositorio de artefactos}

Entendemos como artefactos

Un buen punto de partida es mantener todo el código en un único repositorio
para evitar problemas de gestión de dependencias.

Despliegues unitarios, es decir entregar todo lo necesario en un único paso.

\subsection{Etapas de la pipeline}




\begin{enumerate}
  \item Pre-commits hacia rama \textbf{develop}.
  \begin{enumerate}
    \item Test unitarios
    \item Compilación de binarios (Android, Windows, Linux)
  \end{enumerate}
  \item Merge de develop hacia deploy
  \item Test
  \item Deploy a las plataformas de distribución.
\end{enumerate}

\subsection{Pipeline}

\begin{enumerate}
  \item Merge hacia la rama \textbf{deploy}.
  \item Test unitarios, de integración y static analysis.
  \item Builds Windows, Android y Linux.
  \item Test de rendimiento
  \item Test de aceptación
  \item Test de sistema (instalación y ejecución)
  \item Deploy a las plataformas de distribución.
\end{enumerate}


% =============================================================================

\subsection{Tipos de Test}\label{pipeline:tipos-de-test}

\begin{itemize}
  \item \textbf{Unit Test:} Test sencillos y específicos hechos para probar
      funciones o partes puntuales del código. Deberían correr en memoria y no
      llamar entidades externas como bases de datos, sistema de archivos,
      acceder a redes, etc. Lo importante es probar la unidad de código y no
      sus dependencias. \item \textbf{Integration Test:} Test que combinan
      secciones unitarias de código para comprobar que la interacción de sus
      distintos métodos funcionan correctamente.
  \item \textbf{Acceptance Test:} Se prueba el software y el sistema al
      completo desde el punto de vista de un agente externo al desarrollo
      revisando que el programa satisface todas las especificaciones del
      usuario. \item \textbf{Performance Test:} Se revisa que el software
      cumpla tiempos de respuesta, estabilidad y \lsc{FPS}.
  \item \textbf{Static Analysis:} 
  \item \textbf{Sign-Offs: Regulaciones:} 
  \item \textbf{Security Test:} 
  \item \textbf{Scalability Test:} 
\end{itemize}


Exportar a Linux, Windows y Android:

Linux:
\begin{lstlisting}
$ godot --no-window --export-debug "Linux" export/linux/Bakumapu.x86_64
\end{lstlisting}

Windows:
\begin{lstlisting}
$ godot --no-window --export-debug "Windows" export/windows/Bakumapu.exe
\end{lstlisting}

Android:
\begin{lstlisting}
$ godot --no-window --export-debug "Android" export/android/Bakumapu.apk
\end{lstlisting}
