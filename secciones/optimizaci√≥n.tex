%%%%%%%%%%%%%%%%%%%%%%%%%%%%%%%%%%%%%%%%%%%%%%%%%%%%%%%%%%
%% Optimización

\section{Optimización}\label{optimizacion:optimizacion}

\todoii{ToDo: Optimización}{Desarrollar optimización}


\subsection{¿Qué optimizar?}
El principio básico es: \emph{no optimizar nada que no necesite optimización}.
Revisar la información del Debugger panel, el monitor de recursos y el resto de opciones de Godot para encontrar los elementos que lastren el rendimiento. En base a esa información se deberán tomar las distintas decisiones 

\subsection{Referencias}
\begin{itemize}
\item \href{https://docs.godotengine.org/en/stable/tutorials/debug/debugger_panel.html}{Debugger panel.}
\item \href{https://docs.godotengine.org/en/stable/getting_started/scripting/gdscript/gdscript_styleguide.html#inferred-types}{Inferred types en Godot.}
\item \href{https://docs.godotengine.org/en/stable/tutorials/plugins/gdnative/gdnative-cpp-example.html}{GDNative C++ example.}
\end{itemize}
