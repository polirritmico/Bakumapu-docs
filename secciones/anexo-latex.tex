%%%%%%%%%%%%%%%%%%%%%%%%%%%%%%%%%%%%%%%%%%%%%%%%%%%%%%%%%%
%% Anexo LaTeX

%\appendix
\section{Comandos \LaTeX}\label{anexo-latex}

Los comandos son palabras claves que el compilador interpreta para generar el documento. Siempre deben comenzar con el signo backslash (\textbackslash).

\noindent Para información con respecto al flujo de trabajo del Documento técnico, revisar el apartado \nameref{flujo:documentacion}.

\noindent A continuación una lista con los comandos más fundamentales:

\begin{enumerate}
\item Comentarios:
\lstset{style=latex}
\begin{lstlisting}
Esto es un texto normal.
% Esto es un comentario. No se compila.
Texto % Comentario.
\end{lstlisting}
\item Escapar símbolos:
\begin{lstlisting}
Necesito escribir una linea \_ y un porcentaje \%
El signo \ funciona con este comando: \textbackslash.
\end{lstlisting}
\item Secciones (en orden descendente):
\begin{lstlisting}
\section{Titulo de seccion}
\subsection{Titulo de subseccion}
\subsection*{Titulo.} % Asterisco para no ennumerar.
\subsubsection{Titulo de subsubseccion}
\paragraph{Titulo de seccion parrafo}
\end{lstlisting}
\item Ajustes tipográficos:
\begin{lstlisting}
Negrita: Texto a \textbf{negrita}.
Enfasis: Texto a \emph{enfasis}
Cursiva: Texto a \textit{cursiva}.
Versalita: Texto a \lsc{versalita}.
\end{lstlisting}

\item Ajustes de párrafo:
	\begin{enumerate}
	\item Alineación:
	\begin{lstlisting}
\begin{centering} % alternativas: flushleft, flushright
Texto alineado en base al valor en begin.
\end{centering}
	\end{lstlisting}
	\Needspace{4\baselineskip}
	\item Saltos de línea:
	\begin{lstlisting}
Los saltos necesitan este comando\\
sino se encadenan en una misma linea.

Los parrafos se separan con una linea en blanco.
	\end{lstlisting}

	\item Quitar sangría:
	\begin{lstlisting}
\noindent Primera linea sin sangria.
	\end{lstlisting}
	\end{enumerate}

\item Referencias:
\begin{lstlisting}
% Lo que queremos referenciar:
\subsection{Titulo subseccion}
% Le agregamos label. Ojala usar {nombre_archivo:descripcion}
\subsection{Titulo subseccion}\label{modelado:titulo-sec}

% Para crear la referencia hacia ese titulo:
La siguiente referencia: \nameref{modelado:titulo-sec}

% Ojo: En label{} solo ASCII basico, no tildes ni enie.
\end{lstlisting}

\item Listas:
\begin{lstlisting}
\begin{enumerate} % Para listas no numeradas usar itemize.
    \item Primer item numerado.
    \item Segundo item numerado.
\end{enumerate}
\end{lstlisting}

\item Imágenes:
\begin{lstlisting}
\begin{figure}[h] % h de here, t top, b bottom o nada.
    \centering
    \includegraphics[]{ruta/al/archivo}
    \caption{Subtitulo.}
    \label{fig:imagen} % Si queremos label para referencia.
\end{figure}
\end{lstlisting}

\Needspace{7\baselineskip}
\item Código:
\begin{lstlisting}
\begin{lstlisting*} % sin el asterisco
% Dentro de lstlisting solo ASCII basico sin tildes ni enies
var test = Clase.llamado(ejemplo.datos, objeto.pos())
$ comando -opciones ruta_a_archivo.txt
\end{lstlisting*} % sin el asterisco
\end{lstlisting}

\item Otros:
\begin{enumerate}
	\item Mover a la siguiente página si no hay espacio disponible:
	\begin{lstlisting}
% el 2\baselineskip es para obtener el alto de 2 lineas
\Needspace{2\baselineskip}
	\end{lstlisting}
	
	\item Salto de página:
	\begin{lstlisting}
\pagebreak
	\end{lstlisting}
	
	\item Espacios indivisible (non-breaking space):
	\begin{lstlisting}
60~hz, 50~km, siglo~\lsc{XVI}.
	\end{lstlisting}
	\lstset{style=bash}
\end{enumerate}

\end{enumerate}