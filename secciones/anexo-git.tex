%%%%%%%%%%%%%%%%%%%%%%%%%%%%%%%%%%%%%%%%%%%%%%%%%%%%%%%%%%
%% Anexo GIT

\section{Comandos de GIT}\label{anexo-git}

Este anexo es una referencia rápida a los comandos y configuraciones más comunes en el trabajo con \lsc{GIT}. Cuando en un comando dice origin es lo mismo que escribir la \lsc{URL} del repo en GitHub.

\noindent Otra referencia rápida que puede ser de utilidad se puede encontrar \href{https://rogerdudler.github.io/git-guide}{aquí}.

\noindent Para información sobre el flujo de trabajo del repositorio ver el apartado \nameref{flujo:repositorio}.

%%%%%%%%%%%%%%%%%%%%%%%%%%%%%%%%%%%%%%%%%%%%%%%%%%%%%%%%%%
%%%%%%%%%%%%%%%%%%%%%%%%%%%%%%%%%%%%%%%%%%%%%%%%%%%%%%%%%%

\subsection{Trabajo local}

\begin{itemize}[label=-]

\item Ver el estado de la rama actual:
\begin{lstlisting}
$ git status
\end{lstlisting}

\item Cambiar a una rama:
\begin{lstlisting}
$ git checkout rama
\end{lstlisting}

\item Crear una nueva rama en base a la rama actual y cambiar a ella:
\begin{lstlisting}
$ git checkout -b nueva_rama
\end{lstlisting}

\item Agregar un archivo nuevo, editado o borrado al área de pruebas:
\begin{lstlisting}
$ git add archivo
\end{lstlisting}

\item Agrega todos los cambios al área de pruebas:
\begin{lstlisting}
$ git add -A
\end{lstlisting}

\item Agrega comentarios al commit en el área de pruebas:
\begin{lstlisting}
$ git commit -m "A description with my changes"
\end{lstlisting}
Para comentarios largos (más de 50 caracteres) usar:
\begin{lstlisting}
$ git commit
\end{lstlisting}
Esto abre \lsc{VIM} en \textit{normal mode} para editar el comentario. Pulsamos la tecla \textbf{i} para entrar al \textit{insert mode}, escribimos los cambios normalmente y cuando terminemos pulsamos la tecla \textbf{esc} para volver al \textit{normal mode} y luego \textbf{ZZ} para guardar y salir. Por el funcionamiento de \lsc{GIT}, el mensaje deberá tener esta estructura:
\begin{itemize}
\item En la primera línea un \lsc{HEADER} de máx. 50 caracteres.
\item Segunda línea en blanco.
\item Las líneas siguientes contienen el body del mensaje con toda la explicación, detalles, listas, etc.
\end{itemize}
\begin{lstlisting}
$ git commit
<Dentro de VIM pulsamos i para ingresar al insert mode>
First line as a HEADER of 50 char or less

Some extended description of all the changes:
- change 1
- change 2
- etc.
<Al terminar el texto, pulsamos la tecla esc y luego ZZ>
\end{lstlisting}

\item Si queremos editar el mensaje del commit previo:
\begin{lstlisting}
$ git commit --amend
\end{lstlisting}

\item Mostrar diff de archivos en el área de staging:
\begin{lstlisting}
$ git diff --staged
\end{lstlisting}

\item Deshacer todos los cambios en base al último commit:
\begin{lstlisting}
$ git restore .
\end{lstlisting}

\item Borrar ramas locales ya eliminadas en origin:
\begin{lstlisting}
$ git fetch -p
$ git remote prune origin
\end{lstlisting}

\item Merges de ramas:
\begin{lstlisting}
# Cambiamos a la rama de destino
$ git checkout rama-destino
# Actualizamos la rama de destino
$ git fetch
$ git pull
# Con todo preparado hacemos el merge
$ git merge rama-con-los-cambios
\end{lstlisting}

\end{itemize}

%%%%%%%%%%%%%%%%%%%%%%%%%%%%%%%%%%%%%%%%%%%%%%%%%%%%%%%%%%
%%%%%%%%%%%%%%%%%%%%%%%%%%%%%%%%%%%%%%%%%%%%%%%%%%%%%%%%%%

\subsection{Acciones con el repositorio remoto}

\begin{itemize}[label=-]

\item Actualizar el repositorio local desde origin:
\begin{lstlisting}
$ git fetch
\end{lstlisting}

\item Actualizar y hacer merge al repo local desde origin:
\begin{lstlisting}
$ git pull
\end{lstlisting}

\item Enviar la rama a origin:
\begin{lstlisting}
$ git push -u origin rama
\end{lstlisting}

\item Envía los cambios locales de la rama actual a Github:
\begin{lstlisting}
$ git push
\end{lstlisting}

\item Borrar ramas:
\begin{lstlisting}
# Borrar la rama local
$ git branch -d rama
# Borrar la rama en el repo remoto
$ git push origin -d rama
\end{lstlisting}
\end{itemize}

%%%%%%%%%%%%%%%%%%%%%%%%%%%%%%%%%%%%%%%%%%%%%%%%%%%%%%%%%%
%%%%%%%%%%%%%%%%%%%%%%%%%%%%%%%%%%%%%%%%%%%%%%%%%%%%%%%%%%

\subsection{Configuración}

\begin{itemize}[label=-]

\item Clonar el repositorio en el espacio de trabajo local:
\begin{lstlisting}
$ git clone https://github.com/polirritmico/Bakumapu
\end{lstlisting}

\item Archivo de configuración de GIT:
\begin{lstlisting}
$ ./config/git/config
\end{lstlisting}

\begin{lstlisting}
conf[user]
name = USUARIO
email = MAIL_REGISTRADO_EN_GITHUB
[credential]
helper = store
[core]
autocrlf = input
\end{lstlisting}

\item Credentials en archivo $\sim$/.config/git/credentials (asociado al usuario y al repositorio)

\item Archivos y directorios ignorados por \lsc{GIT} (en .gitignore dentro de Bakumapu):
\begin{lstlisting}
# Godot-specific ignores
.import/
export/
export.cfg
export_presets.cfg

# Mono-specific ignores
.mono/
data_*/
\end{lstlisting}

\Needspace{7\baselineskip}

\item Evitar problemas de formato de archivo entre Windows (\lsc{CRLF}) y Linux (\lsc{LF}):
\begin{lstlisting}
$ .gitattributes:
\end{lstlisting}
\noindent\begin{minipage}{.45\textwidth}
\begin{lstlisting}
# Set the default behavior, in
# case people
# don't have core.autocrlf set

* text eol=lf

# Explicitly declare text
# files you want to always
# be normalized and converted 
# to native line endings on
# checkout.

*.godot text
*.tscn text
*.gd text
*.tres text
*.import text
*.md text
*.txt text
*.json text
*.xml text
*.py text
*.c text
*.h text

# binary files that should not
# be modified

# fonts
*.ttf binary
*.otf binary
\end{lstlisting}
\end{minipage}\hfill
\begin{minipage}{.45\textwidth}
\begin{lstlisting}[escapechar=\%]
# images

*.png binary
*.jpg binary
*.jpeg binary
*.webp binary
*.aseprite binary
*.gif binary
*.xcf binary
*.svg binary
*.kra binary

# sound

*.wav binary
*.ogg binary
*.sf2 binary
*.midi binary
*.amr binary
*.musx binary
*.mp3 binary

# misc

*.zip binary
*.rar binary
*.tar.gz



%
\end{lstlisting}
\end{minipage}

\end{itemize}
