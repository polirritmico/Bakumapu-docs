%%%%%%%%%%%%%%%%%%%%%%%%%%%%%%%%%%%%%%%%%%%%%%%%%%%%%%%%%%
%% Anexo GIT

\section{Comandos de GIT}\label{anexo-git}

Este apartado es una referencia rápida a los comandos y configuraciones más comunes en el trabajo con \lsc{GIT}.

\noindent Otra referencia rápida que puede ser de utilidad se puede encontrar \href{https://rogerdudler.github.io/git-guide}{aquí}.

\noindent Para información sobre el flujo de trabajo del repositorio ver el apartado \nameref{flujo:repositorio}.

\begin{itemize}[label=-]
	\item Configuración:
	\begin{lstlisting}
$ ./config/git/config
	\end{lstlisting}
	
	\begin{lstlisting}
conf[user]
name = USUARIO
email = MAIL_REGISTRADO_EN_GITHUB
[credential]
helper = store
[core]
autocrlf = input
	\end{lstlisting}
	
	\item Credentials en archivo $\sim$/.config/git/credentials (asociado al usuario y al repositorio)
	
	\item Archivos y directorios ignorados por \lsc{GIT} (en .gitignore dentro de Bakumapu):
	\begin{lstlisting}
# Godot-specific ignores
.import/
export/
export.cfg
export_presets.cfg

# Mono-specific ignores
.mono/
data_*/
	\end{lstlisting}
	
	\Needspace{7\baselineskip}
	
	\item Evitar problemas de formato de archivo entre Windows (\lsc{CRLF}) y el resto de sistemas:
	\begin{lstlisting}
$ .gitattributes:
	\end{lstlisting}
	\noindent\begin{minipage}{.45\textwidth}
	\begin{lstlisting}
# Set the default behavior, in
# case people
# don't have core.autocrlf set

* text eol=lf

# Explicitly declare text
# files you want to always
# be normalized and converted 
# to native line endings on
# checkout.

*.godot text
*.tscn text
*.gd text
*.tres text
*.import text
*.md text
*.txt text
*.json text
*.xml text
*.py text
*.c text
*.h text

# binary files that should not
# be modified

# fonts
*.ttf binary
*.otf binary
	\end{lstlisting}
	\end{minipage}\hfill
	\begin{minipage}{.45\textwidth}
	\begin{lstlisting}[escapechar=\%]
# images

*.png binary
*.jpg binary
*.jpeg binary
*.webp binary
*.aseprite binary
*.gif binary
*.xcf binary
*.svg binary
*.kra binary

# sound

*.wav binary
*.ogg binary
*.sf2 binary
*.midi binary
*.amr binary
*.musx binary
*.mp3 binary

# misc

*.zip binary
*.rar binary
*.tar.gz



%
	\end{lstlisting}
	\end{minipage}

	\item Clonar el repositorio en el espacio de trabajo local:
	\begin{lstlisting}
$ git clone https://github.com/polirritmico/Bakumapu
	\end{lstlisting}

	\item Ver el estado de la rama actual:
	\begin{lstlisting}
$ git status
	\end{lstlisting}
	
	\item Cambiar a una rama (por ejemplo develop):
	\begin{lstlisting}
$ git checkout develop
	\end{lstlisting}
	
	\item Crear una nueva rama en base a la rama actual (ejemplo “ft-01”:
	\begin{lstlisting}
$ git checkout -b ft-01
	\end{lstlisting}
	
	\item Borrar ramas 
	\begin{lstlisting}
$ git branch -d rama
$ git push origin -d rama
	\end{lstlisting}
	
	\item Borrar ramas locales ya eliminadas en el repo
	\begin{lstlisting}
$ git fetch -p
$ git remote prune origin
	\end{lstlisting}
	
	\item Merges de ramas 
	\begin{lstlisting}
$ git checkout rama-destino
$ git fetch
$ git pull
$ git merge rama-con-codigo-a-mergear
	\end{lstlisting}
	
	\item Enviar la rama local a Github (origin es un alias de la ruta al repo en github):
	\begin{lstlisting}
$ git push -u origin ft-01
	\end{lstlisting}
	
	\item Agrega todos los cambios al área de pruebas:
	\begin{lstlisting}
$ git add .
	\end{lstlisting}
	
	\item Agrega comentarios al commit:
	\begin{lstlisting}
$ git commit -m descripcion
	\end{lstlisting}
	
	\item Envía los cambios locales de la rama actual a Github:
	\begin{lstlisting}
$ git push
	\end{lstlisting}
	
	\item Actualizar el repositorio local desde Github:
	\begin{lstlisting}
$ git pull
	\end{lstlisting}
\end{itemize}

