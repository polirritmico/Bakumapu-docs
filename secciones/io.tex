%%%%%%%%%%%%%%%%%%%%%%%%%%%%%%%%%%%%%%%%%%%%%%%%%%%%%%%%%%%
% IO

\section{Input/Output.}\label{io:input-output}

\todoii{ToDo: Desarrollar}{Desarrollar sección IO}

El sistema interactuará con el usuario a través de inputs y outputs específicos. A futuro dentro del desarrollo será preciso implementar algunas tecnologías en concreto. De momento solo es necesario mantener la flexibilidad del código y estudiar la propia implementación para decidir con mejores indicadores. Las decisiones al respecto serán de responsabilidad del equipo de diseño en mutuo acuerdo con el equipo técnico.

En cualquier caso, a continuación se presenta un listado con elementos recurrentes y algunos que desde ya se deberían considerar:

\subsection{Inputs.}
Son básicamente 3 elementos los que interactúan con el sistema a nivel de usuario: Archivos de carga, guardado y de configuración, periféricos como teclado, ratón, gamepad o touchscreen; e internet para actualizaciones y servicios en la nube como savegames y configuraciones.

A nivel interno, estos inputs se refieren a los archivos de configuración, stats, niveles, diálogos, sprites, audio, etc.

\subsubsection{Archivos.}


Para los jugadores se refieren básicamente a savegames y archivos de configuración.

\subsubsection{Controles.}
\begin{description}
\item[Teclado:] Además

\item[Gamepad:] description

\item[Mouse:] description

\item[Touchpad:] description

\item[Touchscreen:] description

\item[Giroscopio:] description

\end{description}

\subsubsection{Internet.}

\subsection{Outputs.}

\subsubsection{Video.}
La proporción a soportar será de 16:9 en resoluciones de 720p y 1080p a 60~hz.

\subsubsection{Audio.}
El audio a soportar será estéreo a 44~khz.

\subsubsection{Vibración.}
Rumble.

\subsubsection{Otros.}