%%%%%%%%%%%%%%%%%%%%%%%%%%%%%%%%%%%%%%%%%%%%%%%%%%%%%%%%%%%
%% Trabajando
%% La idea es ir tomando las notas acá mientras
%% desarrollamos el código y luego traspasamos las
%% anotaciones ese texto a la sección que corresponda.

\section{Trabajando...}\label{Trabajando}

\subsection{Level Manager}

src/levels/template/template\_lvl.tscn Archivo de nivel base usado para el test

src/test/unit/test\_level\_manager.gd Script con el test de carga

src/test/unit/level\_manager/level.tscn Archivo de test creado desde el nivel base para el test unitario.

El test inicia LevelManager, usa el método loadLevel() y usa los getters de LvlMng para extraer la data. Luego compara los datos obtenidos con los esperados.

%%\begin{center}***\end{center}

\subsection{Documentación}

Pasar feature branch a rama develop

Underscore \_ a funciones privadas

\subsection{Modelación UI}

Dialogue\_manager dentro de la escena de diálogos, DialogueUI, en CanvasLayer

Todo lo relacionado a UI es muy propenso a cambio

No referenciar directamente con \$ sino usar un export(NotePath) var \_nodo

¿Singleton GameEvents para desacoplar GM?

Las clases interesadas en las señales se connectan en el ready hacia GameEvents. Por ejemplo, GameEvents tiene la señal "pelota\_chocando", y dentro del objeto arco en ready conectamos la señal al método pelota\_en\_arco. 

\subsection{Dialogue\_manager}

Dialogos en formato \textbf{.tres}

\subsection{Pipeline}

Ajustar export name a version
