%%%%%%%%%%%%%%%%%%%%%%%%%%%%%%%%%%%%%%%%%%%%%%%%%%%%%%%%%%
%% Flujo de despliegue

\section{Entrega continua}\label{cd:entrega-continua}

\subsection{Pipeline o Flujo de entrega}\label{cd:flujo-de-entrega}

La idea de una pipeline (tubería) es simplemente el conjunto de herramientas y
procedimientos necesarios para transformar un commit a un output publicable.

Su trabajo es determinar si el software se encuentra según nuestros criterios
en un estado publicable o no. Si el software necesita ciertos mínimos de
rendimiento, entonces nuestro flujo de entrega deberá incluir automatizaciones
de test de rendimiento (fps mínimos por ejemplo).

El sistema nos permite determinar desde el primer momento los criterios de
lanzamiento y organizar todo el flujo de trabajo en pos de esos criterios.

\subsection{Tipos de Test}

- Unit Test
- Acceptance Test
- Performance Test
- Static Analysis
- Sign-Offs: Regulaciones
- Security Test
- Scalability Test

\subsection{Eficiencia}

Un concepto importante con respecto a nuestro pipeline es la eficiencia del
sistema, lo que queremos lograr es ir desde los commits a un resultado lanzable
múltiples veces al día. Por lo mismo es muy importante resolver el cómo
optimizar el proceso y todas las validaciones necesarias para corresponder a
los criterios que queremos que conciernan a nuestro pipeline.

Un buen punto de partida es intentar lograr un ciclo completo del flujo de entrega en menos de 1 hora, para mantener margen suficiente para corregir cualquier problema que ocasione un commit.

\subsection{Punto de Partida}

Un buen punto de partida es mantener todo el código en un único repositorio para evitar problemas de gestión de dependencias.

Despliegues unitarios, es decir entregar todo lo necesario en un único paso.




