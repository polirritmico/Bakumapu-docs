%%%%%%%%%%%%%%%%%%%%%%%%%%%%%%%%%%%%%%%%%%%%%%%%%%%%%%%%%%%
%% Flujo de trabajo

\section{Flujo de trabajo y control de versiones}\label{flujo:flujo-de-trabajo}

\subsection{Metodología Kanban}\label{flujo:metodologia-kanban}
\subsubsection{El Tablero Kanban}\label{flujo:tablero-kanban}
Trello será la herramienta principal para mantener la coordinación del flujo de trabajo a través de un \textbf{Tablero Kanban}. Para el desarrollo del proyecto dividiremos el trabajo en tareas y las iremos incorporando a este tablero dentro de columnas que vienen a representar nuestros procesos de producción.

La idea es anotar cada tarea del proyecto en una tarjeta y agregarla a la columna de la izquierda. Mientras se trabaja en la tarea, su tarjeta va avanzando de izquierda a derecha hasta llegar a la última columna.  

\begin{figure}[h]
	\centering
	\includegraphics[width=\textwidth]{imágenes/tablero01.png}
	\caption{Tablero Kanban en Trello.}
\end{figure}

\subsubsection{Límites de trabajo y columnas asociadas}\label{flujo:limites-de-trabajo}
Cada columna tiene asociado un límite máximo de tareas simultáneas (\lsc{WIP}), por lo que si el límite ha sido alcanzado no se podrán incorporar nuevas tareas a la columna hasta haber avanzado una de sus tarjetas a la columna siguiente. Este límite se puede ver en el número dentro de llaves [~].

Al mismo tiempo cada columna tendrá una división vertical, dejando a la izquierda las tareas en proceso y a la derecha las tareas que ya han completado la etapa. El límite es compartido por ambas secciones por lo que independiente de la sección en la que se encuentre la tarjeta, no puede haber más tarjetas que el límite establecido en la columna. Ver \autoref{flujo:figura-col}.

\begin{figure}[H]
	\centering
	\caption{WIP y división de columnas.}
	\label{flujo:figura-col}
	\includegraphics[width=0.75\textwidth]{imágenes/tablero_col.png}
\end{figure}

Lamentablemente la plataforma Trello no permite dividir una misma columna en dos secciones, por lo que se utilizarán dos columnas distintas para un mismo proceso que llamaremos columnas asociadas. Para identificarlas se enumerarán todas las columnas y las asociadas mantendrán el mismo número. Por ejemplo en la \autoref{flujo:figura-wip}, la columna “[5] Trabajando” ha sido dividida en las columnas “2. [5] Trabajando” y “2. Terminadas”. Ya que ambas comienzan con “2.” sabemos que se refieren al mismo proceso y que sumando el total de tareas entre ambas no debemos superar las [5] (4 + 1 en este caso).

\begin{figure}[h]
	\centering
	\caption{Límites de trabajo en Trello.}
	\label{flujo:figura-wip}
	\includegraphics[width=\textwidth]{imágenes/tablero02.png}
\end{figure}

Respetar los límites de cada columna facilitará la distribución de trabajo y nos permitirá tomar decisiones más acertadas y realistas al planificar.

\subsubsection{Descripción de las columnas del tablero}\label{flujo:descripcion-de-columnas}
A continuación las columnas del tablero en orden de izquierda a derecha:

\begin{enumerate}
	\item \textbf{0. Tareas:} Primera columna; acá dejamos las tareas en sus tarjetas y las ordenamos según prioridad (arriba las más urgentes). Solo se debería empezar a trabajar en la tarea superior. La prioridad de las tareas será determinada por el equipo.
	
	\item \textbf{1. Desglose de tareas:} Las tareas que entran a esta columna se analizan y dividen en tareas acotadas. Es \textit{muy importante} que las tareas sean descompuestas en tareas sencillas para que no tomen más de unos cuantos días en completarse.
	
	\item \textbf{1. Ajuste de tarjeta:} A cada tarjeta en esta columna habrá que asignarle una ID (revisar \nameref{flujo:nombres-de-tarjetas}, ajustar su nombre y descripción de modo que quede perfectamente claro de qué se trata. Si es necesario, añadimos información relevante en la tarjeta.

	\item \textbf{1. Asignación:} Aquí la tarea ya ajustada espera a que alguien la tome y se asigne a ella. Este encargado será responsable de completar la tarea hasta la columna \textbf{Implementadas}.
	
	\item \textbf{2. Trabajando - Test:} En esta etapa escribimos el test y verificamos que falle (más información en el apartado \nameref{principios:TDD}).
	
	\item \textbf{2. Trabajando - Código:} En esta etapa se implementa la funcionalidad hasta que se hayan pasado todos los test correspondientes y haya concluido el ciclo de refactorización con los test aprobados. Una vez hecho el merge hacia develop se pasa la tarjeta a la siguiente columna.
	
	\item \textbf{2. Implementadas:} Acá quedan las tarjetas con tareas terminadas esperando (idealmente) la revisión de otro desarrollador, quien se deberá asignar a la tarjeta antes de moverla a la siguiente etapa.
	
	\item \textbf{3. En revisión:} Acá van las tareas que se estén revisando. Si tienen problemas o bugs pequeños se resuelven aquí mismo, si tuviera bugs o un problema más grande podría volver a analizarse el desglose con alta prioridad (Columna 1. Desglose de tareas).
	
	\item \textbf{3. Finalizadas:} Aquí van las tareas terminadas. Se archivan al hacer el merge de un conjunto de features completo a main (cuando sube la versión).
\end{enumerate}

\subsubsection{Flexibilidad del tablero}\label{flujo:flexibilidad}
El tablero es una herramienta que permite visualizar las distintas áreas del proyecto, identificar cuellos de botella y contribuir a la gestión mediante límites de trabajo; no obstante, no se debe olvidar que \emph{es solo una guía}. Si el flujo de trabajo cambia o el tablero entorpece en algo, se debe ajustar rápidamente.

\subsubsection{Nombres de Tarjetas e ID}\label{flujo:nombres-de-tarjetas}
Los nombres de las tareas se deben corresponder con la tarea o funcionalidad a implementar y a su rama asociada en \lsc{GIT}. Más detalles en el apartado \nameref{organizacion:nombres-de-ramas}.

%%%%%%%%%%%%%%%%%%%%%%%%%%%%%%%%%%%%%%%%%%%%%%%%%%%%%%%%%%%
%%%%%%%%%%%%%%%%%%%%%%%%%%%%%%%%%%%%%%%%%%%%%%%%%%%%%%%%%%%

%\pagebreak
\subsection{Repositorio}\label{flujo:repositorio}
El repositorio \lsc{GIT} de Bakumapu se encuentra alojado en la siguiente dirección: \url{https://github.com/polirritmico/Bakumapu}. Todos los participantes deberán contar con una cuenta con permisos de acceso y contribución.

\subsubsection{Modelo de Integración Continua}\label{flujo:modelo-de-ramas}
La contribución y el despliegue del código será organizada a través del modelo de Integración continua (\href{https://en.wikipedia.org/wiki/Continuous_integration}{Continuous Integration}). Este modelo implica que \emph{solo hay una versión del código que importa: la actual}. Para ello se trabajará principalmente en una misma rama que mantendrá una única versión del código de desarrollo. Esto implica que no podemos hacer commits solo cuando la feature esté completamente implementada, sino que debemos hacer commits continuos a medida que vamos trabajando en la implementación (idealmente varios commits en la misma jornada de trabajo). De ahí la importancia de tener un sistema robusto de test unitarios, de modo que todo commit haya pasado todos los test locales.

Más información al respecto se puede encontrar en el apartado \ref{despliegue:despliegue-del-codigo}.

\begin{figure}[H]
	\centering
	\includegraphics[width=\textwidth]{imágenes/git.png}
	\caption{Ramas en \lsc{GIT}.}
\end{figure}

%\Needspace{3\baselineskip}
\noindent\textbf{Ramas principales.}\label{flujo:ramas-principales}

\noindent Bakumapu contará con una rama principal \textbf{develop} y 2 ramas de producción \textbf{main} y \textbf{release}:

\begin{enumerate}
	\renewcommand{\labelenumi}{\alph{enumi}.}
	\item \textbf{develop:} Es la rama principal del desarrollo ya que todo el código se escribirá permanentemente en ella. La idea es que cada desarrollador vaya haciendo continuos commits hacia \textbf{develop} de tal forma que los cambios en el código sean visibles para el resto del equipo y la implementación de una nueva funcionalidad pase el menor tiempo posible fuera de la rama. De este modo se facilitará la prevención temprana de bugs y problemas complejos de integración.
	
	\item \textbf{main:} En esta rama se tratará de mantener el código lo más estable posible. Representa la última entrega del desarrollo y por lo tanto se creará a partir de \textbf{develop} cuando el código haya completado el conjunto de features de cada etapa de implementación del proyecto. Los testers trabajarán en base a esta rama.
	
	\item \textbf{release:} Se creará a partir de \textbf{main} una vez que esta haya sido extensamente probada y depurada. Esta rama será la que pasará el código a producción por eso debe ser la más robusta (por ejemplo deploys automáticos a todas las copias instaladas y conectadas a internet).
\end{enumerate}

\paragraph{Ramas de implementación}\label{pg:ramas-de-implementacion}

Pese a que la mayoría de la implementación debería ser trabajada directamente en \textbf{develop}, para contadas excepciones se podría necesitar el uso de ramas separadas:

\begin{enumerate}
	\renewcommand{\labelenumi}{\alph{enumi}.}
	\item \textbf{Ramas de features:} Debe crearse una para cada funcionalidad que se quiera implementar a partir del estado actual de la rama \textbf{develop}. El nombre de la rama debe comenzar con “\textbf{ft-}” y corresponderse con el de su tarjeta asociada en el tablero (apartado \nameref{flujo:nombres-de-tarjetas}). Se debe tratar de acortar lo más posible la existencia de esta rama de modo que si en determinado punto es posible hacer un merge hacia develop, se realice y se termine la implementación de vuelta en la rama principal.
	
	\item \textbf{Ramas de bugs:} El nombre de la rama debe comenzar con “\textbf{bug-}” y corresponderse con el de su tarjeta asociada en el tablero (apartado \nameref{organizacion:nombres-de-ramas}). Al igual que en el caso de las ramas de features, la idea es trabajar el menor tiempo posible fuera de \textbf{develop}.
\end{enumerate}

\subsubsection{Commits}
Todos los commits deberán tener al inicio del mensaje del commit la \lsc{ID} de la tarea correspondiente (ver el apartado \nameref{organizacion:id-ramas-tarjetas}).

\subsubsection{Uso de GIT}
Revisar el anexo \nameref{anexo-git}.

%%%%%%%%%%%%%%%%%%%%%%%%%%%%%%%%%%%%%%%%%%%%%%%%%%%%%%%%%%%
%%%%%%%%%%%%%%%%%%%%%%%%%%%%%%%%%%%%%%%%%%%%%%%%%%%%%%%%%%%

\subsection{Documentación}\label{flujo:documentacion}

\subsubsection{El documento de Diseño técnico}\label{flujo:documento-de-diseno}
A nivel de desarrollo muchas veces terminamos dedicando más tiempo a estudiar el código y a entender su funcionamiento que a escribir nueva funcionalidad, por ello la documentación se vuelve tan relevante. Un código más fácil de entender ahorra tiempo, por eso ayuda a mejorar la estabilidad del software y en general todo el desarrollo se torna más productivo. La documentación técnica de Bakumapu estará dividida en este documento y en los archivos de código (apartado \nameref{flujo:documentacion-en-codigo}).

El presente texto, como ya se ha mencionado, tiene como objetivo desempeñar tres funciones principales:
\begin{enumerate}[noitemsep]
	\item Usarse como referencia ante dudas técnicas o de modelado.
	
	\item Servir de instrumento de diseño.
	
	\item Entregar toda la información relevante acerca del flujo de trabajo y del funcionamiento del software para integrar a nuevos miembros del equipo.
\end{enumerate}

Para que estos objetivos se cumplan, el documento debe mantenerse actualizado a medida que se vaya escribiendo el código y tomando las distintas decisiones de diseño e implementación. Por lo mismo se ha generado un repositorio especial para ello (apartado \nameref{flujo:repositorio-de-documentacion}). 

\subsubsection*{¿Qué se documenta aquí?}
No se debe confundir la documentación de este texto con los comentarios o explicaciones dentro del código. El código debe estar debidamente documentado dentro de los archivos y líneas correspondientes (apartado \nameref{flujo:documentacion-en-codigo}). No obstante, cuando haya modificaciones importantes que involucren cambiar o definir la interacción entre clases o elementos de ámbitos más globales, se deberán anotar en este texto a modo de referencia técnica en el apartado \nameref{modelado:funcionamiento-general}.

\paragraph*{Importante:}
Es muy relevante dejar en claro que el objetivo no es escribir una explicación \emph{línea a línea} de cómo funciona el código, sino una \emph{noción general} de ámbitos o lógicas más globales. Con señalar el sentido de estas entidades dentro del sistema y su interacción con el resto será suficiente.

\subsubsection{Modificando el documento}\label{flujo:modificando-el-documento}
Para modificar el documento, basta con utilizar un editor de textos sencillos y seguir la nomenclatura del sistema \LaTeX. En el apartado \nameref{flujo:latex} se proporcionará una breve guía con un listado de los comandos relevantes. En cualquier caso, considerando que el grueso del documento ya está definido, simplemente es cosa de usar los mismos apartados del documento como ejemplo.

Dado que en ciertos contextos la instalación de \LaTeX\ puede ser bastante engorrosa (la instalación manual de muchísimos paquetes), no es necesario compilar una nueva versión con cada cambio sino simplemente mantener los archivos \lsc{TEX} y la versión dentro de \textbf{Makefile} actualizados.

\subsubsection{Versionado del documento}\label{flujo:versionado-del-documento}
Cada modificación a este documento deberá aumentar la numeración de la subversión en 1 (v0.0\textbf{.1} a v0.0\textbf{.2}). Los primeros 2 índices (v\textbf{0.0.}1) estarán en línea con la última rama \textbf{main} del repositorio. Cada vez que se suba una nueva versión de la rama, se deberá chequear que el documento contenga los cambios relevantes a esa versión e incorporarlos. Con cada cambio de versión menor, la subversión debe volver a cero (v0.\textbf{5.36} a v0.\textbf{6.0}). El versionado de las ramas se discutirá en el apartado \nameref{organizacion:ramas-principales}.

Para ajustar la versión del documento solo hay que editar el archivo \textbf{Makefile}, que se encargará automáticamente de hacer todos los ajustes durante la recompilación:
\begin{center}
\texttt{Versión actual: v\docversion.}
\end{center}

\Needspace{3\baselineskip}
\begin{lstlisting}
Bakumapu-docs $ vim Makefile
\end{lstlisting}
\begin{lstlisting}[language=bash]
SHELL = /bin/sh
# Actualizar con cada cambio
VERSION = 0.0.1
\end{lstlisting}

\subsubsection{Exportar a HTML}\label{flujo:exportar-a-html}
Para exportar a \lsc{HTML} bastará con usar el paquete \textbf{make4ht} (utiliza pdflatex y htlatex). Las instrucciones de compilación están configuradas en el fichero Makefile, por lo que la conversión se automatiza con el comando:
\begin{lstlisting}
Bakumapu-docs $ make html
\end{lstlisting}
Esto generará la página \lsc{HTML} con todos los archivos \lsc{CSS}, de fuentes y de imágenes relevantes en la carpeta \textbf{/docs}. Luego restaría simplemente actualizar el repositorio.

\subsubsection{Repositorio de documentación}\label{flujo:repositorio-de-documentacion}
Un simple repositorio \lsc{GIT} en Gitbub, ubicado en: \url{https://github.com/polirritmico/Bakumapu-docs}. Desde aquí solo se puede revisar el código fuente del \lsc{HTML}, para acceder a una versión renderizada está la siguiente \lsc{URL}: \url{https://polirritmico.github.io/Bakumapu-docs/}.

Para sincronizar el servidor además de los comandos \lsc{GIT} habituales, el archivo \textbf{Makefile} tiene las instrucciones para automatizar el proceso:
\begin{lstlisting}
Bakumapu-docs $ make sync
\end{lstlisting}

%%%%%%%%%%%%%%%%%%%%%%%%%%%%%%%%%%%%%%%%%%%%%%%%%%%%%%%%%%%
%%%%%%%%%%%%%%%%%%%%%%%%%%%%%%%%%%%%%%%%%%%%%%%%%%%%%%%%%%%

\subsection{Formato y documentación dentro del código}\label{flujo:documentacion-en-codigo}
La búsqueda de simplicidad en el diseño también aplica a la documentación del código. Idealmente éste debe estar “autodocumentado”, es decir que los nombres de las variables, métodos y clases den cuenta de manera transparente e intuitiva su rol dentro de la lógica del algoritmo. En los casos más complejos, es de vital importancia añadir comentarios no solo para facilitar la comprensión de líneas más complejas, sino para ayudar al futuro proceso de refactorización y debbuging. \emph{Los nombres largos no lastran la eficiencia del código}.

En cuanto al formato del código dado que \lsc{GDS}cript está basado en Python, además de las propias \href{https://docs.godotengine.org/en/stable/getting_started/scripting/gdscript/gdscript_styleguide.html}{sugerencias de Godot} se recomienda seguir la \href{https://www.python.org/dev/peps/pep-0008/}{guía de estilo PEP-8}. En especial las siguientes indicaciones:

\begin{itemize}[noitemsep]
	\item Límite horizontal de 79 caracteres. 
	\item Separación de 1 línea en blanco entre funciones y 2 entre clases.
	\item Indentación por 4 espacios.
	\item Operadores y variables separados por un espacio:
	\begin{lstlisting}
var ejemplo = Vector2(2, 5 + PI.get(2))
	\end{lstlisting}
\end{itemize}

%%%%%%%%%%%%%%%%%%%%%%%%%%%%%%%%%%%%%%%%%%%%%%%%%%%%%%%%%%%
%%%%%%%%%%%%%%%%%%%%%%%%%%%%%%%%%%%%%%%%%%%%%%%%%%%%%%%%%%%

\subsection{Google Drive}\label{flujo:google-drive}
Se manejará la \href{https://drive.google.com/open?id=1p8u-1UpXts8OHGRHEZLSIiQrqqx0Y4Kt}{carpeta compartida Bakumapu}, cuyo acceso será proporcionado a todos los miembros del desarrollo. En la raíz de esta carpeta se encuentran los documentos principales del diseño del juego, y las siguientes subcarpetas:

\begin{itemize}

	\item \textbf{Diálogos:} Contendrá planillas con datos que serán importados a Godot programáticamente, es decir se deberá desarrollar un script o programa que transforme su contenido a \lsc{XML}, \lsc{CSV} o \lsc{JSON} y este sea manejable por Godot con \emph{muy poca} o nada de intervención.

	\item \textbf{Gameplay:} Contiene el árbol de decisiones del gameplay y la información de las distintas quests.
	
	\item \textbf{Herramientas:} Contiene los instaladores, código fuente, o links de descarga del software mencionado en el apartado \nameref{intro:software-y-herramientas} además de los scripts de desarrollo. También contiene tutoriales para los colaboradores no técnicos del proyecto.
	
	\item \textbf{Locaciones:} Mapas y descripciones de los distintos lugares del Bakumapu.
	
	\item \textbf{Referencias:} Libros, imágenes, documentos, audios, videos y todo material referenciado para el desarrollo, inspiración, discusión o diseño del juego.

	Más información de estas herramientas en los apartados \nameref{kit:cutscenes-y-dialogos} y \nameref{kit:quests}. Además, estos archivos deberán seguir la convención de nombres detallada en el apartado \nameref{organizacion:nombres-de-archivos}.
	
	\item \textbf{Personajes:} Fichas de personajes.
	
\end{itemize}

%%%%%%%%%%%%%%%%%%%%%%%%%%%%%%%%%%%%%%%%%%%%%%%%%%%%%%%%%%%
%%%%%%%%%%%%%%%%%%%%%%%%%%%%%%%%%%%%%%%%%%%%%%%%%%%%%%%%%%%

\subsection{LaTeX}\label{flujo:latex}
Pese a lo que pueda parecer, el uso de \LaTeX\ a nivel básico es bastante sencillo, simplemente es escribir el texto y utilizar comandos que definan el contenido para el compilador. En términos prácticos, básicamente es delimitar una sección, subsección o sub-subsección con el comando correspondiente y separar los párrafos con una línea en blanco entremedio.

En cualquier caso, los mismos archivos \lsc{TEX} del documento son un buen modelo de referencia. Si se quiere agregar información o apuntes para facilitar el uso de \LaTeX\ se debería agregar en este apartado. En lo posible tratar de no agregar nuevos paquetes.

Para una lista de los comandos más comunes y una rápida explicación al respecto, revisar el anexo \nameref{anexo-latex}.
